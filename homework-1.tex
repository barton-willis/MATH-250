\documentclass[12pt,fleqn]{exam}
\usepackage{amsmath,fleqn, url}
\usepackage{pifont,enumerate,color}
%\pagestyle{empty}
%\usepackage[euler-digits,euler-hat-accent,T1]{eulervm}
\usepackage{fourier}

\newcommand{\reals}{\mathbf{R}}
\newcommand{\integers}{\mathbf{Z}}

\newenvironment{alphalist}{
  \begin{enumerate}[(a)]
    \addtolength{\itemsep}{-1.0\itemsep}}
  {\end{enumerate}}


\usepackage[activate={true,nocompatibility},final,tracking=true,kerning=true,spacing=true,factor=1100,stretch=10,shrink=10]{microtype}
\usepackage[american]{babel}
\usepackage[T1]{fontenc}
\usepackage{fourier}


\usepackage{array}
\newcolumntype{C}{>$c<$}

\addpoints
\boxedpoints
\pointsinmargin
\pointname{pts}
\begin{document}
\large
\begin{flushleft}
{\bf 
Foundations of Mathematics \\
Homework Assignment 1 \\
Due Saturday 29 January \the\year}
\end{flushleft}

\noindent \emph{This assignment has questions 1 through \numquestions \/ with a total
of \numpoints\/ points.  The box to the left of each question gives the point value for the question. You may hand write this assignment, but you must turn it into Canvas electronically. }




\begin{questions}

\question [2] Give two original examples (differ from the examples given class or from the textbook) of sentences that are  assertions (also known as a statement).
\begin{solution}
  
  \begin{enumerate}
     \item 
     \item 
  \end{enumerate}
\end{solution}
\question [2]  Give two original examples (differ from the examples given class or from the textbook) of sentences that are not assertions.
\begin{solution}
    \begin{enumerate}
     \item 
     \item 
  \end{enumerate}
\end{solution}
\question [2] Let \(P\) be a statement.  Show that \(\lnot (P \land \lnot P)\) is a tautology. Do this by constructing a truth table.
\begin{solution} 


\end{solution}

\question [2] Let \(P\) and \(Q\) be a assertions.  Show that \( (P \implies Q) \equiv (P \implies Q) \)  is not a tautology. Do this by constructing a truth table.

\question  [2] Let $A,B,$ and $C$ be assertions. Show that 
\[ ( A \implies B)\land (B\implies C)) \implies (A \implies C) 
\]
 is a tautology.

\end{questions}
\end{document}

\question [5] From page 15  (\S 1.3), work exercise \textbf{3}.

\begin{solution} We have
\begin{tabular}{|c|c|c|}
\hline
\textbf{P} & \textbf{Q} & \(\mathbf{P} \veebar \mathbf{Q} \) \\ \hline
F & F & F\\
F & T & T\\
T & F & T \\
T & T & F\\
\hline
\end{tabular}.

\textbf{TE} I've seen \(\veebar\) used for an exclusive or in other
texts, but I'm not sure that it's part of the ISO-31 standard
(\url{http://en.wikipedia.org/wiki/ISO_31-11}). And by the way: Our
textbook uses \(\sim \) for logical negation, but the standard is
\(\lnot\).
\end{solution}

\question [5] From page 17  (\S 1.3), work exercise \textbf{11a}.

\begin{solution} (11a) Since \(\mathbf{Q} \vee \sim \mathbf{Q} \) is a tautology, a truth table reduces to two rows:

\begin{tabular}{|c|c|c|}
\hline
\textbf{P} & \(\mathbf{Q} \vee \sim \mathbf{Q} \) & \(P \Rightarrow  \mathbf{Q} \vee \sim \mathbf{Q} \) \\ \hline
F & T & T \\
T & T & T \\

\hline
\end{tabular}.

\textbf{TE} We've shown that \(P \Rightarrow  \mathbf{Q} \vee \sim \mathbf{Q} \) is a tautology.

\end{solution}

\question [5] From page 17  (\S 1.3), work exercise \textbf{11d}.

\begin{solution} (11d) 

\begin{tabular}{|c|c|c|c|c|} \hline
P & Q & P \(\Rightarrow \) Q  & \( \sim Q  \Rightarrow   \sim P \) & \( P \Rightarrow  Q \iff \sim Q  \Rightarrow   \sim P \) \\ \hline 
F & F & T & T & T \\
F & T & T & T & T \\
T & F & F & F & T \\
F & F & T & T & T \\
\hline
\end{tabular}.


\end{solution}

\question [5] A \emph{counterexample} is a \emph{specific example} that shows that a statement is false.
Give a \emph{counterexample} for the following:

\textbf{Proposition} If a function \(F\) is continuous at zero, then \(F\) is differentiable at zero.

\begin{solution} 
\textbf{Counterexample} The absolute value function is continuous but not differentiable at zero.
\end{solution}

\question [5] Show that the operator \(\Rightarrow \) is not associative. Thus, for statements \(P,Q,\) and \(R\), show that 
\(P \Rightarrow  \left(Q \Rightarrow  R\right)\) is \emph{not} logically equivalent to \mbox{\(\left(P \Rightarrow  Q \right) \Rightarrow  R\).} 
You could construct an eight row truth table, but you could also just provide one counterexample.
Of course, to find the one counterexample you might need to check all
eight cases (unless you get lucky or have some insight into the problem.)

\begin{solution}
 We have \(F \Rightarrow  (F \Rightarrow  F) = F \Rightarrow  T = T\), but  \((F \Rightarrow  F) \Rightarrow  F =
  T \Rightarrow  F = F\). Thus   \(\Rightarrow \) is not associative.
\end{solution}



\end{questions}

\end{document}

\question Without using the word `not,' write a \emph{negation} of each statement.
For the negation of `quick,' use the word `slow.' 

\begin{parts}

\part [5] All border collies are quick.

\begin{solution} Some border collies are slow.

\textbf{TE} A counterexample to the statement  ``All border collies are quick''  would be a \emph{specific} example of a
border collie who is slow: For example ``Cousin Elizabeth's border collie Larry is slow'' would
be a counterexample. More generally, there is a counterexample provided there is at least one border collie that is slow.
That is, the negation of ``All border collies are quick'' is ``Some border collies are slow.''


\end{solution}

\part [5] Some border collies are quick.


\begin{solution} All border collies are slow. \end{solution}

\end{parts}

\end{questions}

\end{document}

\question [10] Show that the operator \(\rightarrow\) is not
associative. Thus, for statements \(P,Q,\) and \(R\), show that \(P
\to \left(Q \to R\right)\) is \emph{not} logically equivalent to
\mbox{\(\left(P \to Q \right) \to R\).} You could construct an eight
row truth table, but you could also just provide one counterexample.
Of course, to find the one counterexample you might need to check all
eight cases (unless you get lucky or have some insight into the problem.)

\begin{solution}
 We have \(F \rightarrow (F \rightarrow F) = F \rightarrow T = T\), but  \((F \rightarrow F) \rightarrow F =
  T \rightarrow F = F\). Thus   \(\rightarrow\) is not associative.
\end{solution}



\question Ternary logic\footnote{Ternary logic is sometimes known as \emph{Kleene logic}.
Stephen Kleene (1909--1994) was an American mathematician. Maybe because he didn't want to 
be known as ``Dr. Clean,'' he pronounced his name \emph{Klaynee.}} 
introduces the additional truth value of \emph{unknown}. Denoting unknown by
U, the truth table for negation, disjunction, conjunction, and logical equivalence are

\begin{tabular}{|c|c|}
\hline
\(\mathbf{P}\) & \(\mathbf{\lnot P}\) \\ \hline
T & F \\
U & U \\
F & T \\
\hline
\end{tabular}, \quad \quad
\begin{tabular}{|c|c|c|c|c|}
\hline
\(\mathbf{P}\) & \(\mathbf{Q} \) & \(\mathbf{P \vee Q}\) &  \(\mathbf{P \wedge Q}\)  &  \(\mathbf{P \leftrightarrow Q}\) \\ \hline
T & T & T & T & T\\
T & U & T & U & F\\
T & F & T  & F & F\\
U & T & T  & U & F\\
U & U & U & U & T \\
U & F & U  & F & F \\
F & T & T & F & F\\
F & U & U & F & F\\
F & F & F & F & T\\ \hline
\end{tabular}.




\begin{parts} 

\part [5] Show that in ternary logic, the negation is an \emph{involution}. Thus show that
\(P\) is logically equivalent to \(\lnot \lnot P\).

\begin{solution} We have
\begin{align*}
  &\lnot \lnot F = \lnot T = T, \\
  &\lnot \lnot U = \lnot U = U, \\
 &\lnot \lnot T = \lnot F = T.
\end{align*}
So negation is an involution.
\end{solution}

\part [5] Assuming that \(P \rightarrow Q \) is logically equivalent
to \(\left (\lnot P\right) \vee Q\), find the truth table for \(P
\rightarrow Q\) in ternary logic.

\begin{solution} A truth table for the Kleene conditional is

\vspace{0.1in}
\begin{tabular}{|c|c|c|c|c|} \hline
\(P\) & \(Q\) & \(\lnot  P\) & \(\lnot P \lor Q\) & \(P \rightarrow Q\)\\  \hline
T & T & F & T & T \\ T & U & F & U & U \\ T & F & F
  & F & F \\ U & T & U & T & T \\ U & U & U & U & U \\ U & F & U & U
  & U \\ F & T & T & T & T \\ F & U & T & T & T \\ F & F & T & T & T \\ \hline
\end{tabular}.

\end{solution}

\part [5] Are the statements \( P \rightarrow Q\) and \(\lnot Q \rightarrow \lnot P\) logically
equivalent?

\begin{solution} Yes,  \( P \rightarrow Q\) and \(\lnot Q \rightarrow \lnot P\) are logically
equivalent. We have
\[
    ( P \rightarrow Q) \leftrightarrow (\lnot P \vee Q) \leftrightarrow (\lnot \lnot Q \vee \lnot P) 
\leftrightarrow (\lnot Q \rightarrow \lnot P).
\]
We've used the facts that negation is an involution and that the disjunction is commutative.

\end{solution}
\part [5] In ternary logic, is \(\lnot \left(P \vee Q \right) \) logically  equivalent to 
\( \lnot \left(\lnot P \wedge \lnot Q \right)\)?

\begin{solution} No, these statements are not logically equivalent; for example
\( \lnot (T \vee T) = F\), but  \(\lnot (\lnot T \wedge \lnot T) = \lnot (F \wedge F) = \lnot F = T.\)

\end{solution}

\end{parts}

\end{questions}
\end{document}



